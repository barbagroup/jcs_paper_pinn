%! TEX root = main.tex

In this study, we aimed to expand upon our previous work \cite{chuang_experience_2022} by exploring the effectiveness of physics-informed neural networks (PINNs) in predicting vortex shedding in a 2D cylinder flow at $Re = 200$.
It should be noted that our focus is limited to forward problems involving non-parameterized, incompressible Navier-Stokes equations.

To ensure the correctness of our results, we verified and validated all involved solvers.
Aside from using as a baseline results obtained with PetIBM, we used three PINN solvers in the case study: a steady data-free PINN, an unsteady data-free PINN, and a data-driven PINN.
Our results indicate that while both data-free PINNs produced steady-state solutions similar to traditional CFD solvers, they failed to predict vortex shedding in unsteady flow situations.
On the other hand, the data-driven PINN predicted vortex shedding only within the timeframe where PetIBM training data were available, and beyond this timeframe the prediction quickly reverted to the steady-state solution.
Additionally, the data-driven PINN showed limited extrapolation capabilities and produced meaningless predictions at unseen coordinates.
Our Koopman analysis suggests that PINN methods may be dissipative and dispersive, which inhibits oscillation and causes the computed flow to return to a steady state.
This analysis is also consistent with the observation of a spectral bias inherent in neural networks \cite{rahaman_spectral_2019}.

For forward problems, data-free PINNs will need more theoretical and mathematical refinement to compete with traditional numerical methods in terms of accuracy and computational cost, making data-driven PINNs more promising due to their capabilities in handling applications that are challenging for traditional methods, such as flow reconstruction or surrogate modeling with sparse data.
Thus, an interesting research question arises: how do data-driven PINNs compare to classical deep learning approaches, where neural networks are not constrained by partial differential equations?
Both data-driven PINNs and classical deep learning approaches work well for interpolation settings but struggle with extrapolation beyond the training data.
In situations where only sparse data are available, data-driven PINNs still work while classical deep learning may not.
However, data-driven PINNs are also more expensive to train and require significantly more computational resources due to the presence of high-order derivatives in the loss function.
It is therefore important to assess the cost-performance ratio of data-driven PINNs when deployed in real-world applications.
One obvious factor affecting this ratio in data-driven PINNs and classical deep learning is the ease of obtaining additional training data.
For example, if training data are obtained from traditional numerical simulations, meaning that high-quality data can be acquired abundantly, then classical deep learning may be a cheaper option than data-driven PINNs.
Thus, it would be valuable to quantitatively assess the benefits of data-driven PINNs compared to classical deep learning and understand the associated cost-performance trade-offs.

% vim:ft=tex