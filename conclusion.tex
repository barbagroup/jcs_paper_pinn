%! TEX root = main.tex

In this study, we aimed to expand upon our previous work \cite{chuang_experience_2022} by exploring the effectiveness of physics-informed neural networks (PINNs) in predicting vortex shedding in a 2D cylinder flow at $Re = 200$.
It should be noted that our focus is limited to forward problems involving non-parameterized, incompressible Navier-Stokes equations.

To ensure the correctness of our results, we verified and validated all involved solvers.
Aside from using as a baseline results obtained with PetIBM, we used three PINN solvers in the case study: a steady data-free PINN, an unsteady data-free PINN, and a data-driven PINN.
Our results indicate that while both data-free PINNs produced steady-state solutions similar to traditional CFD solvers, they failed to predict vortex shedding in unsteady flow situations.
On the other hand, the data-driven PINN predicted vortex shedding only within the timeframe where PetIBM training data were available, and beyond this timeframe the prediction quickly reverted to the steady-state solution.
Additionally, the data-driven PINN showed limited extrapolation capabilities and produced meaningless predictions at unseen coordinates.
Our Koopman analysis suggests that PINN methods may be dissipative and dispersive, which inhibits oscillation and causes the computed flow to return to a steady state.
This analysis is also consistent with the observation of a spectral bias inherent in neural networks \cite{rahaman_spectral_2019}.

One interesting research question that arises from our findings is how the cost-performance ratio of data-driven PINNs compares to classical deep learning approaches.
While data-free PINNs are commonly considered as numerical methods for solving PDEs, data-driven PINNs are more akin to supervised machine/deep learning.
However, data-free PINNs have been shown to have inferior cost-performance ratios compared to traditional numerical methods for PDEs (in terms of forward and non-parameterized problems).
The literature suggests that PINNs are best utilized in a data-driven configuration, rather than data-free settings.
Therefore, it would be valuable to quantitatively compare the benefits of data-driven PINNs to those of classical deep learning approaches and understand the associated cost-performance trade-offs.

% vim:ft=tex