%! TEX root = main.tex

In recent years, research interest surrounding Physics-Informed Neural Networks (PINNs) has surged.
The idea of using neural networks to represent solutions of ordinary and partial differential equations goes back to the 1990s \cite{dissanayake_neural-network-based_1994,lagaris_artificial_1998}, but upon the term PINN being coined about five years ago, the field exploded. 
Partly, it reflects the immense popularity of all things machine learning and artificial intelligence (ML/AI). 
It also is very attractive to be able to solve differential equations without meshing the domain, and without having to discretize the equations in space and time. 
PINNs are easily implemented in a few lines of code, taking advantage of the ML frameworks that have become available in recent years, such as PyTorch. 
In contrast, traditional numerical solvers for PDEs such as the Navier-Stokes equations can require years of expertise and thousands of lines of code to develop, test and maintain. 
The general optimism in this field has perhaps held back critical examinations of the limitations of PINNs, and the challenges of using them in practical applications. 
Besides, it is well known that the academic literature is biased to positive results, and negative results are rarely published. 

In this paper, we examine the solution of Navier-Stokes equations using PINNs in flows with instabilities, particularly vortex shedding. 
Fluid dynamic instabilities are ubiquitous in nature and engineering applications, and any method arising as a competitor to traditional CFD should be able to handle them. 
In a previous conference paper, we already reported on our observations of the limitations of PINNs in this context \cite{chuang_predicting_2020}. 
Although the solution of a laminar flow with vorticity, the classical Taylor-Green vortex, was well represented by a PINN solver, the same network architecture failed in a flow with vortex shedding. 
The PINN solver reverted to the steady state solution in two-dimensional flow past a circular cylinder at $Re=200$. 
Here, investigate this failure in more detail, comparing with a traditional CFD solver and with a data-driven PINN that receives as training data the solution of the CFD solver. 
We look at various fluid diagnostics, and also use dynamic mode decomposition (DMD) to analyze the flow and help explain the difficulty of the PINN solver to capture oscillatory solutions.

Our PINN solver was built using the software library \emph{Modulus} by NVIDIA, a high-level package built on PyTorch.
For the traditional CFD solver, we used our own code, \emph{PetIBM}, which is open-source and available on GitHub \cite{chuang_petibm_2018}.