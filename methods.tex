%! TEX root = main.tex

We will be solving the 2D incompressible Navier-Stokes equations in primitive-variable form:
\begin{empheq}[left=\left\{\,, right=\right.]{equation}\label{eq:orig-ns}
    \begin{aligned}
    &\nabla \cdot \vec{u} = 0 \\
    &\pdiff{\vec{u}}{t} + \left(\vec{u} \cdot \nabla\right) \vec{u}
        =
        -\frac{1}{\rho}\nabla p + \nu \nabla^2 \vec{u}
    \end{aligned}
\end{empheq}
\noindent Here, $\vec{u} \equiv \left[ u \enspace v \right]^\mathsf{T}$, $p$, $\nu$, and $\rho$ denote the velocity vector, pressure, kinematic viscosity, and the density, respectively.
Let $\vec{x} \equiv \left[ x \enspace y \right]^\mathsf{T} \in \Omega$ and $t \in \left[0,\enspace T\right]$ denote the spatial and temporal domains.
The velocity $\vec{u}$ and pressure $p$ are functions of $\vec{x}$ and $t$ for given fluid properties $\rho$ and $\nu$.
The solution to the Navier-Stokes equations is subject to initial conditions $\vec{u}(\vec{x}, t) = \left[ u_0(\vec{x}) \enspace v_0(\vec{x}) \right]^\mathsf{T}$ and $p(\vec{x}, t) = p_0(\vec{x})$ for $\vec{x} \in \Omega$ and $t=0$.
The Dirichlet boundary conditions are $u(\vec{x}, t) = u_D(\vec{x}, t)$ and $v(\vec{x}, t) = v_D(\vec{x}, t)$, on domain boundaries $\vec{x} \in \Gamma_{\displaystyle u_D}$ and $\Gamma_{\displaystyle v_D}$, respectively.
The Neumann boundary conditions are $\pdiff{u}{\vec{n}}(\vec{x}, t)=u_N(\vec{x}, t)$ and $\pdiff{v}{\vec{n}}=v_N(\vec{x}, t)$, defined on boundaries $\vec{x} \in \Gamma_{\displaystyle u_N}$ and $\Gamma_{\displaystyle v_N}$ correspondingly.
Note that in incompressible flow pressure is a Lagrangian multiplier to enforce the divergence-free condition and does not need boundary conditions theoretically.

\begin{figure*}
    \centering
    \normalsize
    \resizebox{\textwidth}{!}{\begin{tikzpicture}
	\begin{pgfonlayer}{nodelayer}
		\node [style=input] (0) at (-22, 10) {};
		\node [style=input] (1) at (-22, 8) {};
		\node [style=input] (2) at (-22, 6) {};
		\node [style=input] (3) at (-22, 4) {};
		\node [style=input] (4) at (-22, 10) {$x$};
		\node [style=input] (5) at (-22, 8) {$y$};
		\node [style=input] (6) at (-22, 6) {$z$};
		\node [style=input] (7) at (-22, 4) {$t$};
		\node [style=param] (8) at (-20, 11) {$\theta_1^1$};
		\node [style=param] (9) at (-20, 9) {$\theta_2^1$};
		\node [style=param] (10) at (-20, 3) {$\theta_{N_1}^1$};
		\node [style=none] (11) at (-20, 6) {$\vdots$};
		\node [style=none] (13) at (-20, 5) {$\vdots$};
		\node [style=none] (14) at (-20, 7) {$\vdots$};
		\node [style=param] (15) at (-18, 11) {$\theta_1^2$};
		\node [style=param] (16) at (-18, 9) {$\theta_2^2$};
		\node [style=param] (17) at (-18, 3) {$\theta_{N_2}^2$};
		\node [style=none] (18) at (-18, 6) {$\vdots$};
		\node [style=none] (19) at (-18, 5) {$\vdots$};
		\node [style=none] (20) at (-18, 7) {$\vdots$};
		\node [style=input] (21) at (-16, 10) {};
		\node [style=input] (22) at (-16, 8) {};
		\node [style=input] (23) at (-16, 6) {};
		\node [style=input] (24) at (-16, 4) {};
		\node [style=input] (25) at (-16, 10) {$u$};
		\node [style=input] (26) at (-16, 8) {$v$};
		\node [style=input] (27) at (-16, 6) {$w$};
		\node [style=input] (28) at (-16, 4) {$p$};
		\node [style=text only] (31) at (-1.5, 11.75) {\relscale{0.5}$\left\{\begin{aligned}
					&\mathbf{u}=\begin{bsmallmatrix}u&v&w\end{bsmallmatrix}^\mathsf{T}\text{; }\mathbf{x}=\begin{bsmallmatrix}x&y&z\end{bsmallmatrix}^\mathsf{T} \\
					&\Theta=\left\{\theta_i^j: i\in 1..N_j, j=1, 2\right\}
				\end{aligned}\right.$};
		\node [style=none] (32) at (-23, 12.75) {};
		\node [style=none] (33) at (-19, 12.75) {};
		\node [style=none] (34) at (-15, 12.75) {};
		\node [style=none] (36) at (-15, 2) {};
		\node [style=none] (37) at (-19, 2) {};
		\node [style=none] (38) at (-23, 2) {};
		\node [style=none] (39) at (-23, 7) {};
		\node [style=none] (40) at (-15, 7) {};
		\node [style=derivative] (41) at (-12.75, 11.5) {$\frac{\partial u}{\partial t}$};
		\node [style=derivative] (42) at (-11.25, 11.5) {$\frac{\partial u}{\partial x}$};
		\node [style=derivative] (43) at (-9.75, 11.5) {$\frac{\partial u}{\partial y}$};
		\node [style=derivative] (44) at (-8.25, 11.5) {$\frac{\partial u}{\partial z}$};
		\node [style=derivative] (46) at (-12.75, 8.5) {$\frac{\partial v}{\partial t}$};
		\node [style=derivative] (47) at (-11.25, 8.5) {$\frac{\partial v}{\partial x}$};
		\node [style=derivative] (48) at (-9.75, 8.5) {$\frac{\partial v}{\partial y}$};
		\node [style=derivative] (49) at (-8.25, 8.5) {$\frac{\partial v}{\partial z}$};
		\node [style=derivative] (50) at (-12.75, 5.5) {$\frac{\partial w}{\partial t}$};
		\node [style=derivative] (51) at (-11.25, 5.5) {$\frac{\partial w}{\partial x}$};
		\node [style=derivative] (52) at (-9.75, 5.5) {$\frac{\partial w}{\partial y}$};
		\node [style=derivative] (53) at (-8.25, 5.5) {$\frac{\partial w}{\partial z}$};
		\node [style=derivative] (54) at (-12, 2.5) {$\frac{\partial p}{\partial x}$};
		\node [style=derivative] (55) at (-10.5, 2.5) {$\frac{\partial p}{\partial y}$};
		\node [style=derivative] (56) at (-9, 2.5) {$\frac{\partial p}{\partial z}$};
		\node [style=derivative] (57) at (-12, 10.5) {$\frac{\partial^2 u}{\partial x^2}$};
		\node [style=derivative] (58) at (-10.5, 10.5) {$\frac{\partial^2 u}{\partial y^2}$};
		\node [style=derivative] (59) at (-9, 10.5) {$\frac{\partial^2 u}{\partial z^2}$};
		\node [style=derivative] (60) at (-12, 7.5) {$\frac{\partial^2 v}{\partial x^2}$};
		\node [style=derivative] (61) at (-10.5, 7.5) {$\frac{\partial^2 v}{\partial y^2}$};
		\node [style=derivative] (62) at (-9, 7.5) {$\frac{\partial^2 v}{\partial z^2}$};
		\node [style=derivative] (63) at (-12, 4.5) {$\frac{\partial^2 w}{\partial x^2}$};
		\node [style=derivative] (64) at (-10.5, 4.5) {$\frac{\partial^2 w}{\partial y^2}$};
		\node [style=derivative] (65) at (-9, 4.5) {$\frac{\partial^2 w}{\partial z^2}$};
		\node [style=none] (66) at (-12.75, 3.25) {};
		\node [style=none] (67) at (-7.25, 3.25) {};
		\node [style=none] (68) at (-7.25, 1.75) {};
		\node [style=none] (69) at (-12.75, 1.75) {};
		\node [style=none] (70) at (-13.5, 6.25) {};
		\node [style=none] (71) at (-7.25, 6.25) {};
		\node [style=none] (72) at (-7.25, 3.75) {};
		\node [style=none] (73) at (-13.5, 3.75) {};
		\node [style=none] (74) at (-13.5, 9.25) {};
		\node [style=none] (75) at (-7.25, 9.25) {};
		\node [style=none] (76) at (-7.25, 6.75) {};
		\node [style=none] (77) at (-13.5, 6.75) {};
		\node [style=none] (78) at (-13.5, 12.25) {};
		\node [style=none] (79) at (-7.25, 12.25) {};
		\node [style=none] (80) at (-7.25, 9.75) {};
		\node [style=none] (81) at (-13.5, 9.75) {};
		\node [style=none] (82) at (-12.75, 2.5) {};
		\node [style=none] (83) at (-7.25, 2.5) {};
		\node [style=none] (84) at (-13.5, 5) {};
		\node [style=none] (85) at (-7.25, 5) {};
		\node [style=none] (86) at (-13.5, 8) {};
		\node [style=none] (87) at (-7.25, 8) {};
		\node [style=none] (88) at (-13.5, 11) {};
		\node [style=none] (89) at (-7.25, 11) {};
		\node [style=loss] (90) at (-1.5, 10.25) {$L_1 = \nabla \cdot \mathbf{u}$};
		\node [style=loss] (91) at (-1.5, 8) {$\begin{bmatrix}L_2 \\ L_3 \\ L_4 \end{bmatrix}
=
\frac{\mathrm{D} \mathbf{u}}{\mathrm{D} t} +
\frac{1}{\rho}\nabla p - \nu\nabla^2\mathbf{u}$};
		\node [style=loss] (93) at (-1.5, 5.75) {$L_5 = \left.\mathbf{u}\right|_{\mathbf{x}\in\Gamma_D}-\mathbf{BC}_{\Gamma_D}$};
		\node [style=loss] (94) at (-1.5, 4) {$L_6=\left(\nabla\mathbf{u}\cdot\mathbf{n}\right)_{\mathbf{x}\in\Gamma_N}-\mathbf{BC}_{\Gamma_N}$};
		\node [style=loss] (95) at (-1.5, 2.25) {$L_7 = \left.\mathbf{u}\right|_{t=0} - \mathbf{IC}$};
		\node [style=none] (97) at (-14, 12.75) {};
		\node [style=none] (98) at (-6.75, 12.75) {};
		\node [style=none] (99) at (-6.75, 1.25) {};
		\node [style=none] (100) at (-14, 1.25) {};
		\node [style=none] (101) at (-10.25, 12.75) {};
		\node [style=none] (102) at (-14, 7.75) {};
		\node [style=none] (103) at (-6.75, 7.75) {};
		\node [style=none] (104) at (-10.25, 1.25) {};
		\node [style=none] (105) at (-5.75, 12.75) {};
		\node [style=none] (106) at (-5.75, 1.25) {};
		\node [style=none] (107) at (-5.75, 7) {};
		\node [style=none] (108) at (2.75, 12.75) {};
		\node [style=none] (109) at (2.75, 1.25) {};
		\node [style=none] (110) at (2.75, 7) {};
		\node [style=none] (111) at (-0.75, 12.75) {};
		\node [style=none] (112) at (-3, 1.25) {};
		\node [style=text only] (113) at (-14.5, 14) {Automatic differentiation};
		\node [style=text only] (114) at (-5.5, 14) {Multiple objectives};
		\node [style=loss] (115) at (-10.25, -0.25) {$\argmin\limits_{\Theta} L(L_1, \cdots, L_7)$};
		\node [style=text only] (116) at (-1.25, -0.25) {Nonlinear optimization};
		\node [style=text only] (117) at (-19.5, 12.25) {$u, v, w, p = f(x, y, z, t; \Theta)$};
	\end{pgfonlayer}
	\begin{pgfonlayer}{edgelayer}
		\draw [style=one arrow] (4) to (8);
		\draw [style=one arrow] (4) to (9);
		\draw [style=one arrow] (4) to (10);
		\draw [style=one arrow] (5) to (8);
		\draw [style=one arrow] (5) to (9);
		\draw [style=one arrow] (5) to (10);
		\draw [style=one arrow] (6) to (8);
		\draw [style=one arrow] (6) to (9);
		\draw [style=one arrow] (6) to (10);
		\draw [style=one arrow] (7) to (8);
		\draw [style=one arrow] (7) to (9);
		\draw [style=one arrow] (7) to (10);
		\draw [in=180, out=0] (8) to (15);
		\draw [style=one arrow] (8) to (16);
		\draw [style=one arrow] (8) to (17);
		\draw [style=one arrow] (9) to (15);
		\draw [style=one arrow] (9) to (16);
		\draw [style=one arrow] (9) to (17);
		\draw [style=one arrow] (10) to (15);
		\draw [style=one arrow] (10) to (16);
		\draw [in=180, out=0] (10) to (17);
		\draw [style=one arrow] (15) to (25);
		\draw [style=one arrow] (15) to (26);
		\draw [style=one arrow] (15) to (27);
		\draw [style=one arrow] (15) to (28);
		\draw [style=one arrow] (25) to (16);
		\draw [style=one arrow] (16) to (26);
		\draw [in=120, out=-60] (16) to (27);
		\draw [style=one arrow] (16) to (28);
		\draw [style=one arrow] (25) to (17);
		\draw [style=one arrow] (17) to (26);
		\draw [style=one arrow] (17) to (27);
		\draw [style=one arrow] (17) to (28);
		\draw [style=frame] (34.center)
			 to (40.center)
			 to (36.center)
			 to (37.center)
			 to (38.center)
			 to (39.center)
			 to (32.center)
			 to [in=180, out=0] (33.center)
			 to cycle;
		\draw [style=node edge] (67.center)
			 to (83.center)
			 to (68.center)
			 to (69.center)
			 to (82.center)
			 to (66.center)
			 to cycle;
		\draw [style=node edge] (88.center)
			 to (78.center)
			 to (79.center)
			 to (89.center)
			 to (80.center)
			 to (81.center)
			 to cycle;
		\draw [style=node edge] (75.center)
			 to (87.center)
			 to (76.center)
			 to (77.center)
			 to (86.center)
			 to (74.center)
			 to cycle;
		\draw [style=node edge] (70.center)
			 to (71.center)
			 to (85.center)
			 to (72.center)
			 to (73.center)
			 to (84.center)
			 to cycle;
		\draw [style=frame] (97.center) to (101.center);
		\draw [style=frame] (101.center) to (98.center);
		\draw [style=frame] (98.center) to (103.center);
		\draw [style=frame] (103.center) to (99.center);
		\draw [style=frame] (99.center) to (104.center);
		\draw [style=frame] (104.center) to (100.center);
		\draw [style=frame] (100.center) to (102.center);
		\draw [style=frame] (102.center) to (97.center);
		\draw [style=frame] (105.center) to (107.center);
		\draw [style=frame] (107.center) to (106.center);
		\draw [style=frame] (108.center) to (110.center);
		\draw [style=frame, in=90, out=-90] (110.center) to (109.center);
		\draw [style=frame] (105.center) to (111.center);
		\draw [style=frame] (111.center) to (108.center);
		\draw [style=frame] (109.center) to (112.center);
		\draw [style=frame] (112.center) to (106.center);
		\draw [style=frame connector, bend left=15] (33.center) to (101.center);
		\draw [style=frame connector, bend left=15] (101.center) to (111.center);
		\draw [style=frame connector, in=0, out=-90, looseness=0.50] (112.center) to (115);
	\end{pgfonlayer}
\end{tikzpicture}
\unskip}
    \caption{
        A graphical illustration of the workflow in PINNs:
        $\vec{x} \equiv \left[ x \enspace y \right]^\mathsf{T} \in \Omega$ and $t \in \left[0,\enspace T\right]$ denote the spatial and temporal domains.
        $\vec{u} \equiv \left[ u \enspace v \right]^\mathsf{T}$, $p$, $\nu$, and $\rho$ represent the velocity vector, pressure, kinematic viscosity, and the density, respectively.
        $G(\vec{x}, t; \theta)$ is a neural network model that approximates the solution to the Navier-Stokes equations with a set of free model parameters denoted by $\theta$.
        The $\left\{h_1^1, \cdots, h_{N_1}^1, \cdots, h_1^\ell, \cdots, h_{N_\ell}^\ell\right\}$, called hidden layers in neural networks, can be deemed as some intermediate values or temporary results during the calculations of the approximate solutions.
        Given spatial-temporal coordinates $(x, y, t)$, the neural network returns an approximate solution $(u, v, p)$ at this coordinate.
        We then apply automatic differentiation to obtain required derivatives.
        With the approximate solutions and the derivatives, we are able to calculate the residuals (also called losses, denoted by symbol $L$) between the approximates and PDEs, as well as the initial and boundary conditions. 
        Using the aggregated squared losses, we can determine the free model parameters $\theta$ by a least-square method.
    }
    \label{fig:pinn-workflow}
\end{figure*}

When using physics-informed neural networks, PINNs, we approximate the solutions to equation \eqref{eq:orig-ns} with a neural network model $G(\vec{x}, t; \theta)$:
\begin{equation}\label{eq:G-network}
    \begin{bmatrix}
        u(\vec{x}, t) \\ v(\vec{x}, t) \\ p(\vec{x}, t)
    \end{bmatrix}
    \approx
    G(\vec{x}, t; \theta),
\end{equation}
where $\theta$ represents a set of free model parameters we need to determine later.
A common choice of $G$ is an MLP (multilayer perceptron) network, which can be represented as follows:

\begin{gather}\label{eq:mlp-formula}
    \vec{h}^0 \equiv \begin{bmatrix} x & y & t \end{bmatrix}^\mathsf{T} \\
    \vec{h}^k =
        \sigma_{k-1}\left(\mat{A}^{k-1}\vec{h}^{k-1}+\vec{b}^{k-1}\right)
        \text{, for } 1 \le k \le \ell \\
    \begin{bmatrix} u & v & p \end{bmatrix}^\mathsf{T}
        \approx
        \vec{h}^{\ell+1} = \sigma_\ell\left(\mat{A}^\ell\vec{h}^\ell+\vec{b}^\ell\right)
\end{gather}
The vectors $\vec{h}^k$ for $1 \le k \le \ell$ are called hidden layers, and carry intermediate evaluations of the transformations that take the input (spatial and temporal variables) to the output (velocity and pressure values).
$\ell$ denotes the number of hidden layers.
The elements in these vectors are called neurons, and $N_k$ for $1 \le k \le \ell$ represents the number of neurons in each hidden layer.
To have a consistent notation, we use $\vec{h}^0$ to denote the vector of the inputs to the model $G$, which contains spatial-temporal coordinates.
Similarly, $\vec{h}^{\ell+1}$ denotes the outputs of $G$, corresponding to the approximate solutions $u$, $v$, and $p$ at every spatial point and time instant. 
$\mat{A}^k$ and $\vec{b}^k$ for $0 \le k \le \ell$ are parameter matrices and vectors holding the free model parameters that will be found via optimization, 
$\theta = \left\{ \mat{A}^0, \vec{b}^0, \cdots, \mat{A}^\ell, \vec{b}^\ell\right\}$.
Finally, $\sigma_k$ for $0 \le k \le \ell$ are vector-valued functions, called activation functions, that are applied element-wise to the vectors $\vec{h}^k$.
In neural networks, the activation functions are responsible for providing the non-linearity in an MLP model.
Throughout this work, we use $\sigma_0 = \cdots = \sigma_\ell = \sigma(\vec{z}) = \frac{\vec{z}}{1 + \exp(\vec{z})}$, the classical sigmoid function.
The parameters $\ell$, $N_k$, and the choices of $\sigma_k$ together control the model complexity of the PINNs that use MLP networks.

As with all other numerical methods for PDEs, the calculations of spatial and temporal derivatives of velocity and pressure play a crucial role.
While a numerical approximation (e.g., finite difference) may be a more robust choice---as seen in early-day literature on neural networks for differential equations \cite{dissanayake_neural-network-based_1994,lagaris_artificial_1998}---, it is common to see the use of automatic differentiation nowadays.
Automatic differentiation is a general technique to find derivatives of a function by decomposing it into elementary functions with a known derivative, and then applying the chain rule of calculus to get exact derivative of the more complex function.
Note that the word {\it exact} here refers to being exact in terms of the model $G$, rather than to the true solution of the Navier-Stokes equations. 
A detailed review of automatic differentiation can be found in reference \cite{griewank_automatic_1988}.
Major deep learning programming libraries, such as TensorFlow and PyTorch, offer the user automatic differentiation features.

Once we have obtained derivatives, we are able to calculate residuals, also called losses in the terminology of machine learning.
As shown in figure \ref{fig:pinn-workflow}, given a spatial-temporal coordinate $(x, y, t)$, we can calculate up to \num{10} loss terms, depending on where in the domain this spatial-temporal point is located. 
Figure \ref{fig:pinn-workflow} is only shown as an illustration of the PINN methodology using the solution workflow specifically for the Navier-Stokes equations \eqref{eq:orig-ns}.
The number and definitions of loss terms may change, for example, when we have some boundary segments with Robin conditions or when we are solving 3D problems.

Finally, we determine the free model parameters using a least-squares optimization, as shown in the last block in figure \ref{fig:pinn-workflow}.
To be specific, in this work we used the Adam stochastic optimization method for this process. 
We first randomly sampled some spatial-temporal collocation points from the computational domain, including all boundaries.
These points are called {\it training points} in the terminology of machine learning.
Depending on where a training point is located in the domain, it may result in multiple loss terms, as described in the previous paragraph.
An aggregated squared loss is obtained over all loss terms of all training points.
In this work, all loss terms were taken to have the same weights.
The Adam optimization then finds the optimal model parameters, i.e., $\theta=\left\{\mat{A}^0, \vec{b}^0, \cdots, \mat{A}^\ell, \vec{b}^\ell\right\}$, based on the gradients of the aggregated loss with respect to model parameters.
In other words, the desired model parameters are those giving the minimal aggregated squared loss.

Note that in figure \ref{fig:pinn-workflow} we consider that if-conditions determine the loss terms to calculate on a training point.
In practice, however, we sample points in subgroups separately from within the domain, on the boundaries, and at $t=0$.
Each subgroup of training points is only responsible for specific loss terms.
We also use a batched approach for the optimization,
meaning that not all training points are used during each individual optimization iteration.
The batched approach only uses a sample of the training points to calculate the losses and the gradients of the aggregated loss in each optimization iteration.
Hereafter, the term {\it training} will be used interchangeably with the optimization process.

In this section, we only introduce the specific details of PINNs required for our work.
References \cite{dissanayake_neural-network-based_1994,lagaris_artificial_1998,cai_physics-informed_2021} provide more details of these methods in general.

% vim:ft=tex: