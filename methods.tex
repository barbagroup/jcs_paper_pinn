%! TEX root = main.tex

This work aims at solving the 2D incompressible Navier-Stokes equations, as show below:
\begin{empheq}[left=\left\{\,, right=\right.]{equation}\label{eq:orig-ns}
    \begin{aligned}
    &\nabla \cdot \vec{u} = 0 \\
    &\pdiff{\vec{u}}{t} + \left(\vec{u} \cdot \nabla\right) \vec{u}
        =
        -\frac{1}{\rho}\nabla p + \nu \nabla^2 \vec{u}
    \end{aligned}
\end{empheq}
where $\vec{u} \equiv \left[ u \enspace v \right]^\mathsf{T}$, $p$, $\nu$, and $\rho$ denote the velocity vector, pressure, kinematic viscosity, and the density, respectively.
Let $\vec{x} \equiv \left[ x \enspace y \right]^\mathsf{T} \in \Omega$ and $t \in \left[0,\enspace T\right]$ denote the spatial and temporal domains.
The velocity $\vec{u}$ and pressure $p$ are functions of $\vec{x}$ and $t$ for a given $\rho$ and $\nu$.
The solution to the Navier-Stokes equations is subject to initial conditions $\vec{u}(\vec{x}, t) = \left[ u_0(\vec{x}) \enspace v_0(\vec{x}) \right]^\mathsf{T}$ and $p(\vec{x}, t) = p_0(\vec{x})$ for $\vec{x} \in \Omega$ and $t=0$.
The Dirichlet boundary conditions are $u(\vec{x}, t) = u_D(\vec{x}, t)$ and $v(\vec{x}, t) = v_D(\vec{x}, t)$, corresponding to boundaries $\vec{x} \in \Gamma_{\displaystyle u_D}$ and $\Gamma_{\displaystyle v_D}$, respectively.
The Neumann boundary conditions are $\pdiff{u}{\vec{n}}(\vec{x}, t)=u_N(\vec{x}, t)$ and $\pdiff{v}{\vec{n}}=v_N(\vec{x}, t)$, defined on boundaries $\vec{x} \in \Gamma_{\displaystyle u_N}$ and $\Gamma_{\displaystyle v_N}$ correspondingly.
Note that in incompressible flow, pressure is a Lagrangian multiplier to enforce the divergence-free condition and does not need boundary conditions theoretically.

\begin{figure*}
    \centering
    \normalsize
    \resizebox{\textwidth}{!}{\begin{tikzpicture}
	% network's frame
	\node [none] (network) at (0, 0) {
		Network
		$\left[\begin{smallmatrix} \vec{u} \\ p \end{smallmatrix}\right] =
		G(\vec{x}, t; \vec{\theta})$
	};

	% network's nodes: the input layer
	\node [below=1.5 of network.south west, input, anchor=north west] (nin1) {$x$};
	\node [below=0.5 of nin1, input] (nin2) {$y$};
	\node [below=0.5 of nin2, input] (nin3) {$t$};

	% network's nodes: the 1st hidden layer
	\node [right=0.3 of nin1, param] (nh12) {$h_2^1$};
	\node [above=0.5 of nh12, param] (nh11) {$h_1^1$};
	\node [below=0.5 of nh12, none]	(nh13) {$\vdots$};
	\node [below=0.5 of nh13, none]	(nh14) {$\vdots$};
	\node [below=0.5 of nh14, param] (nh15) {$h_{N_1}^1$};

	% network's nodes: the skipped layer
	\node [above right=0.3 of nh12, none]	(nskip1) {$\cdots$};
	\node [right=0.3 of nh13, none]	(nskip2) {$\cdots$};
	\node [above right=0.3 of nh15, none]	(nskip3) {$\cdots$};

	% network's nodes: the last hidden layer
	\node [right=0.6 of nh12, param] (nh22) {$h_2^\ell$};
	\node [above=0.5 of nh22, param] (nh21) {$h_1^\ell$};
	\node [below=0.5 of nh22, none]	(nh23) {$\vdots$};
	\node [below=0.5 of nh23, none]	(nh24) {$\vdots$};
	\node [below=0.5 of nh24, param] (nh25) {$h_{N_\ell}^\ell$};

	% network's nodes: the output layer
	\node [right=0.5 of nh22, input] (nout1) {$u$};
	\node [below=0.5 of nout1, input] (nout2) {$v$};
	\node [below=0.5 of nout2, input] (nout3) {$p$};

	% network's outer frame
	\node [draw=black!50, fit={(network) (nin2) (nh15) (nh25) (nout2)}] (nnframe){};

	% u derivative nodes
	\node [above right=1.6 and 0.8 of nout1.north east, anchor=south west, input] (dudt) {
		$\frac{\partial u}{\partial t}$};
	\node [right=0.1 of dudt.east, anchor=west, input] (dudx) {
		$\frac{\partial u}{\partial x}$};
	\node [right=0.1 of dudx.east, anchor=west, input] (dudy) {
		$\frac{\partial u}{\partial y}$};
	\node [right=0.1 of dudy.east, anchor=west, input] (d2udx2) {
		$\frac{\partial^2 u}{\partial x^2}$};
	\node [right=0.1 of d2udx2.east, anchor=west, input] (d2udy2) {
		$\frac{\partial^2 u}{\partial y^2}$};

	% u derivatives' outer frame
	\node [draw=black!50, fit={(dudt) (d2udy2)}] (dubox) {};

	% v derivative nodes
	\node [below right=1.2 and 0.8 of nout3.south east, anchor=north west, input] (dvdt) {
		$\frac{\partial v}{\partial t}$};
	\node [right=0.1 of dvdt.east, anchor=west, input] (dvdx) {
		$\frac{\partial v}{\partial x}$};
	\node [right=0.1 of dvdx.east, anchor=west, input] (dvdy) {
		$\frac{\partial v}{\partial y}$};
	\node [right=0.1 of dvdy.east, anchor=west, input] (d2vdx2) {
		$\frac{\partial^2 v}{\partial x^2}$};
	\node [right=0.1 of d2vdx2.east, anchor=west, input] (d2vdy2) {
		$\frac{\partial^2 v}{\partial y^2}$};

	% v derivatives' outer frame
	\node [draw=black!50, fit={(dvdt) (d2vdy2)}] (dvbox) {};

	% p derivative nodes
	\node [above=0.5 of dvdt.north west, anchor=south west, input] (dpdx) {
		$\frac{\partial p}{\partial x}$};
	\node [right=0.1 of dpdx.east, anchor=west, input] (dpdy) {
		$\frac{\partial p}{\partial y}$};

	% p derivatives' outer frame
	\node [draw=black!50, fit={(dpdx) (dpdy)}] (dpbox) {};

	% all derivatives' outer frame
	\node [draw=black!50, fit={(dubox) (dvbox)}] (dervbox) {};

	% loss 5: IC of v (rendered first bc it's at the center)
	\node [right=4.75 of nnframe.east, anchor=south west, input] (loss5) {$
		L_5 = v - v_0
		\text{\enspace if } t = 0
	$};

	% loss 4: IC of u
	\node [above=0.2 of loss5.north west, anchor=south west, input] (loss4) {$
		L_4 = u - u_0
		\text{\enspace if } t = 0
	$};

	% loss 3: momentum y
	\node [above=0.2 of loss4.north west, anchor=south west, input] (loss3) {$
		L_3 = 
			\frac{\partial v}{\partial t} +
			\vec{u} \cdot \nabla v +
			\frac{1}{\rho}\frac{\partial p}{\partial y} -
			\nu \nabla^2 v
		\text{\enspace if } \vec{x} \in {\Omega}
	$};

	% loss 2: momentum x
	\node [above=0.2 of loss3.north west, anchor=south west, input] (loss2) {$
		L_2 = 
			\frac{\partial u}{\partial t} +
			\vec{u} \cdot \nabla u +
			\frac{1}{\rho}\frac{\partial p}{\partial x} -
			\nu \nabla^2 u
		\text{\enspace if } \vec{x} \in {\Omega}
	$};

	% loss 1: continuity
	\node [above=0.2 of loss2.north west, anchor=south west, input] (loss1) {$
		L_1 = \nabla \cdot \vec{u} \text{\enspace if } \vec{x} \in {\Omega}
	$};

	% loss 6: IC of p
	\node [below=0.2 of loss5.south west, anchor=north west, input] (loss6) {$
		L_6 = p - p_0
		\text{\enspace if } t = 0
	$};

	% loss 7: dirichlet bc of u
	\node [below=0.2 of loss6.south west, anchor=north west, input] (loss7) {$
		L_7 = u - u_D \text{\enspace if } \vec{x}\in\Gamma_{\displaystyle u_D}
	$};

	% loss 8: dirichlet bc of v
	\node [below=0.2 of loss7.south west, anchor=north west, input] (loss8) {$
		L_8 = v - v_D \text{\enspace if } \vec{x}\in\Gamma_{\displaystyle v_D}
	$};

	% loss 9: neumann bc of u
	\node [below=0.2 of loss8.south west, anchor=north west, input] (loss9) {$
		L_9 = \frac{\partial u}{\partial \vec{n}} - u_N
		\text{\enspace if } \vec{x}\in\Gamma_{\displaystyle u_N}
	$};

	% loss 10: neumann bc of v
	\node [below=0.2 of loss9.south west, anchor=north west, input] (loss10) {$
		L_{10} = \frac{\partial v}{\partial \vec{n}} - v_N
		\text{\enspace if } \vec{x}\in\Gamma_{\displaystyle v_N}
	$};

	% losses' outer frame
	\node [draw=black!50, fit={(loss1) (loss2) (loss3) (loss10)}] (lossframe){};

	% arg min
	\node [right=0.5 of lossframe.east, anchor=west, input] (argmin) {$
		\argmin\limits_{\theta \in \Theta}
		\sum\limits_{\substack{\vec{x} \in \Omega \cup \Gamma \\ t \in T}}
		\sum\limits_{j=1}^{10} L_j^2
	$};
	\node [above=0.1 of argmin.north, anchor=south, none] (argmintxt) {Optimizing/training};

	% link network's inputs to the 1st hidden layer
	\draw [style=one arrow] (nin1) to (nh11);
	\draw [style=one arrow] (nin1) to (nh12);
	\draw [style=one arrow] (nin1) to (nh15);
	\draw [style=one arrow] (nin2) to (nh11);
	\draw [style=one arrow] (nin2) to (nh12);
	\draw [style=one arrow] (nin2) to (nh15);
	\draw [style=one arrow] (nin3) to (nh11);
	\draw [style=one arrow] (nin3) to (nh12);
	\draw [style=one arrow] (nin3) to (nh15);

	% link network's last hidden layer to the outputs
	\draw [style=one arrow] (nh21) to (nout1);
	\draw [style=one arrow] (nh21) to (nout2);
	\draw [style=one arrow] (nh21) to (nout3);
	\draw [style=one arrow] (nh22) to (nout1);
	\draw [style=one arrow] (nh22) to (nout2);
	\draw [style=one arrow] (nh22) to (nout3);
	\draw [style=one arrow] (nh25) to (nout1);
	\draw [style=one arrow] (nh25) to (nout2);
	\draw [style=one arrow] (nh25) to (nout3);

	% link network's outputs to derivatives
	\draw [style=one arrow] (nout1.east) to (dubox.west);
	\draw [style=one arrow] (nout2.east) to (dvbox.west);
	\draw [style=one arrow] (nout3.east) to (dpbox.west);

	% links to continuity loss
	\draw [style=one arrow] (dubox.east) to (loss1.west);
	\draw [style=one arrow] (dvbox.east) to (loss1.west);

	% links to x momemtum loss
	\draw [style=one arrow] (nout1.east) to (loss2.west);
	\draw [style=one arrow] (nout2.east) to (loss2.west);
	\draw [style=one arrow] (dubox.east) to (loss2.west);
	\draw [style=one arrow] (dvbox.east) to (loss2.west);
	\draw [style=one arrow] (dpbox.east) to (loss2.west);

	% links to y momemtum loss
	\draw [style=one arrow] (nout1.east) to (loss3.west);
	\draw [style=one arrow] (nout2.east) to (loss3.west);
	\draw [style=one arrow] (dubox.east) to (loss3.west);
	\draw [style=one arrow] (dvbox.east) to (loss3.west);
	\draw [style=one arrow] (dpbox.east) to (loss3.west);

	% links to u IC
	\draw[style=one arrow] (nout1.east) to (loss4.west);

	% links to v IC
	\draw[style=one arrow] (nout2.east) to (loss5.west);

	% links to p IC
	\draw[style=one arrow] (nout3.east) to (loss6.west);

	% links to u velocity Dirichlet BC
	\draw [style=one arrow] (nout1.east) to (loss7.west);

	% links to v velocity Dirichlet BC
	\draw [style=one arrow] (nout2.east) to (loss8.west);

	% links to u velocity Neumann BC
	\draw [style=one arrow] (dubox.east) to (loss9.west);

	% links to v velocity Neumann BC
	\draw [style=one arrow] (dvbox.east) to (loss10.west);

	% links from losses to argmin
	\draw [style=one arrow] (lossframe.east) to (argmin.west);

	% denoting automatic derivation
	\node [above right=0.8 and 0.3 of nnframe.north, anchor=west, none] (adtxt) {
		Automatic differentiation
	};
	\draw [-, draw=black!50] (nnframe.north) |- (adtxt.west);
	\draw [-{Latex[length=4]}, draw=black!50] (adtxt.east) -| (dervbox.north);

\end{tikzpicture}
% vim:ft=tex:\unskip}
    \caption{
        A graphical demonstration of the workflow in PINNs.
        $\vec{x} \equiv \left[ x \enspace y \right]^\mathsf{T} \in \Omega$ and $t \in \left[0,\enspace T\right]$ denote the spatial and temporal domains.
        $\vec{u} \equiv \left[ u \enspace v \right]^\mathsf{T}$, $p$, $\nu$, and $\rho$ represent the velocity vector, pressure, kinematic viscosity, and the density, respectively.
        $G(\vec{x}, t; \theta)$ is a neural network model that approximates the solution to the Navier-Stokes equations with a set of free model parameters denoted by $\theta$.
        $\left\{h_1^1, \cdots, h_{N_1}^1, \cdots, h_1^\ell, \cdots, h_{N_\ell}^\ell\right\}$, formally called hidden layers in neural networks, can be deemed as some intermediate values or temporary results during the calculations of the approximate solutions.
        Given a spatial-temporal coordinates $(x, y, t)$, the neural network returns the approximate solution $(u, v, p)$ at this coordinate.
        We then apply automatic differentiation to obtain required derivatives.
        With the approximate solutions and the derivatives, we are able to calculate the residuals (also called losses, denoted by symbol $L$) between the approximates and PDEs, as well as the initial and boundary conditions. 
        Using the aggregated squared losses, we can determine the free model parameters $\theta$ by a least-square fashion.
    }
    \label{fig:pinn-workflow}
\end{figure*}

In PINNs, we approximate the solutions to equation \eqref{eq:orig-ns} with a neural network model $G(\vec{x}, t; \theta)$:
\begin{equation}\label{eq:G-network}
    \begin{bmatrix}
        u(\vec{x}, t) \\ v(\vec{x}, t) \\ p(\vec{x}, t)
    \end{bmatrix}
    \approx
    G(\vec{x}, t; \theta)
\end{equation}
where $\theta$ represents a set of free model parameters we need to determine later.
A common choice of $G$ is an MLP (multilayer perceptron) network:
\begin{gather}\label{eq:mlp-formula}
    \vec{h}^0 \equiv \begin{bmatrix} x & y & t \end{bmatrix}^\mathsf{T} \\
    \vec{h}^k =
        \sigma_{k-1}\left(\mat{A}^{k-1}\vec{h}^{k-1}+\vec{b}^{k-1}\right)
        \text{, for } 1 \le k \le \ell \\
    \begin{bmatrix} u & v & p \end{bmatrix}^\mathsf{T}
        \approx
        \vec{h}^{\ell+1} = \sigma_\ell\left(\mat{A}^\ell\vec{h}^\ell+\vec{b}^\ell\right)
\end{gather}
The vectors $\vec{h}^k$ for $1 \le k \le \ell$, which carry intermediate calculation results, are called hidden layers.
$\ell$ denotes the number of hidden layers.
The elements in these vectors are called neurons, and $N_k$ for $1 \le k \le \ell$ represents the number of neurons in each hidden layer.
To have a consistent notation, we use $\vec{h}^0$ to denote the vector of the input to the model $G$, which needs spatial-temporal coordinates as the inputs.
As well, $\vec{h}^{\ell+1}$ denotes the outputs of $G$, which are the approximate solutions $u$, $v$, and $p$. 
$\mat{A}^k$ and $\vec{b}^k$ for $0 \le k \le \ell$ are parameter matrices and vectors holding free model parameters.
In other words, $\theta = \left\{ \mat{A}^0, \vec{b}^0, \cdots, \mat{A}^\ell, \vec{b}^\ell\right\}$.
Finally, $\sigma_k$ for $0 \le k \le \ell$ are vector-valued functions.
They are called activation functions in neural networks and are responsible for the non-linearity in an MLP model.
Throughout this work, we use $\sigma_0 = \cdots = \sigma_\ell = \sigma(\vec{z}) = \frac{\vec{z}}{1 + \exp(\vec{z})}$.
$\ell$, $N_k$, and the choices of $\sigma_k$ control the model complexity of the PINNs that use MLP networks.

As all other numerical methods for PDEs, the calculations of spatial and temporal derivatives of velocity and pressure play a crucial role.
While a numerical approximation (e.g., finite difference) may be a more robust choice (as seen in early-day literature \cite{dissanayake_neural-network-based_1994,lagaris_artificial_1998}), it is common to see the use of automatic differentiation nowadays.
Automatic differentiation is an algorithm that applies the chain rule of calculus to get exact derivatives.
Note that the word {\it exact} here means being exact in terms of the model $G$, rather than to the true solution of the Navier-Stokes equations. 
A detailed review of automatic differentiation can be found in reference \cite{griewank_automatic_1988}.
Major deep learning programming libraries, such as TensorFlow and PyTorch, have implemented automatic differentiation.

Once we have derivatives, we are able to calculate residuals (or losses in the terminology of machine learning).
As shown in figure \ref{fig:pinn-workflow}, given a spatial-temporal coordinate $(x, y, t)$, we can calculate up to \num{10} loss terms, depending on where in the domain this spatial-temporal point is located. 
Figure \ref{fig:pinn-workflow} is only shown as a demonstration using the solution workflow specifically for the Navier-Stokes equations described in the first paragraph in this section.
The number and definitions of loss terms may change, for example, when we have some boundary segments with Robin conditions or when we are solving 3D problems.

Finally, we determine the free model parameters using a least-square optimization, as shown in the last block in figure \ref{fig:pinn-workflow}.
To be more specific, we used the Adam optimization for this process. 
We first randomly sample some spatial-temporal points from the computational domain, including all boundaries.
These points are called {\it training points} in the terminology of machine learning.
Depending on where a training point is located in the domain, the point may result in multiple loss terms, as described in the previous paragraph.
An aggregated squared loss is obtained over all loss terms of all training points.
In this specific work, all loss terms have the same weights.
The Adam optimization then finds the optimal model parameters (i.e., $\theta=\left\{\mat{A}^0, \vec{b}^0, \cdots, \mat{A}^\ell, \vec{b}^\ell\right\}$) based on the gradients of the aggregated loss with respect to model parameters.
In other words, the desired model parameters are those giving the minimal aggregated squared loss.

Note that, in figure \ref{fig:pinn-workflow}, we use if-conditions to determine which loss terms to calculate regarding a training point.
In practice, however, we sample points into subgroups separately from within the domain, on the boundaries, and at $t=0$.
Then each subgroup of training points is only responsible for known loss terms.
We also use a batched approach for the optimization.
That is, not all training points are used during each individual optimization iteration.
The batched approach only uses a portion of the training points to calculate the losses and the gradients of the aggregated loss in each optimization iteration.
Finally, the term {\it training} will be used interchangeably with the optimization process.

In this section, we only introduce the specific details of PINNs required for our work.
References \cite{dissanayake_neural-network-based_1994,lagaris_artificial_1998,cai_physics-informed_2021} provide more details.

% vim:ft=tex: